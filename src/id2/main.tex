\documentclass[12pt]{report}
   \usepackage[fleqn]{amsmath}
   \usepackage{hyperref}       
   \begin{document}
       \title{Problem ID 2}
       \author{David Andreoletti}
       \date{August 2013} 
       \maketitle       
O(1) solution :)\\ \\
\href{http://en.wikipedia.org/wiki/Fibonacci_number}{Fibonacci Sequence} is 
\begin{equation} 
F_n = F_{n-1} + F_{n-2} 
\end{equation} 
1)\\ \\
First terms in the sequence are:
\begin{align*}
F_n    &= 0, 1, 2, 3, 5, 8, 13, 21, 34, 55 ... \\
n     &= 0, 1, 2, 3, 4, 5, 6,  7,  8,  9  ... \\
\end{align*}
\begin{align*}
G_m    &= 0, 2, 8, 34 ... \\
n     &=  0,  2,  5,  8  ... \\
m     &=  0,  1,  2,  3  ... \\
\end{align*}
\(G_m\) looks like a sequence of even Fibonaci terms. Is \(G_m = A*G_{m-1} + B*G_{m-2} = 2k\)
 with \(m \geq 2\) and \(n \geq 3\) ?  \\ \\
2) \\ \\
Expressing Gm with Fn terms 
\begin{equation} 
\begin{aligned}
 G_m &= C*G_{m-1} + D*G_{m-2} \\
 F_{n+6} &= C*F_{n+3} + D*F_n \\
 F_n &= (F_{n+6} - C*F_{n+3}) / D \\
\end{aligned}
\end{equation} 
We know: \(F_n = F_{n-1} + F_{n-2}\) 
\begin{equation} 
 (F_{n+6} - C*F_{n+3}) / D = F_{n-1} + F_{n-2}
\end{equation} 
We know: \(F_0 = 0, F_1 = 1\) \\ \\
Solve (A) for \(n=5:\) 
\begin{equation} 
\begin{aligned} \label{eq:a1}
  (F_{11} - C*F_8) / D &= F_4 + F_3 \\
  (89 - C*21) / D &= 3 + 2  \\
  89 - 21C - 5D &= 0        \\
\end{aligned}
\end{equation} 
Solve (A) for \(n=8\): 
\begin{equation} \label{eq:a2}
\begin{aligned}
  (F_{14} - C*F_11) / D &= F_7 + F_6 \\
  377 - 89C - 21D &= 0         \\
\end{aligned}
\end{equation} 
Solve \href{https://en.wikipedia.org/wiki/System_of_linear_equations}{System of linear equations}  \eqref{eq:a1} \eqref{eq:a2} to find C and D. \\ \\
\(C = 4, D = 1\) \\ \\
So: \\
\begin{equation} 
\begin{aligned} \label{eq:a3}
 F_{n+6} &= 4*F_{n+3} + F_n     \\
 F_{n+6} &=  0*F_{n+5} + 0*F_{n+4} + 4*F_{n+3} + 0*F_{n+2} + 0*F_{n+1} + F_n \\
 G_m &= 4*G_{m-1} + 1*G_{m-2} \\
\end{aligned}
\end{equation} 
\eqref{eq:a3} is a \href{http://en.wikipedia.org/wiki/Recurrence_relation#Linear_homogeneous_recurrence_relations_with_constant_coefficients}{Linear homogeneous recurrence relations with constant coefficients} of order d=2 (LRS). LRS yields characteristic polynomial of the form: 
\begin{equation} 
   p(t) = t^d - c_1t^d-1 - c_2t^d-2 + ... + c_d
\end{equation} 
 Hence \eqref{eq:a3} 's characteristic polynomial is:
\begin{equation} 
   p(t) = t^2 - 4t - 1
\end{equation} 
Solve p(t) = 0 
\begin{equation}
\begin{aligned} \label{eq:a4}
   0 &= t^2 - 4t - 1 = p(t) \\
     &= (t-2)^2 -4t - 5           \text{since (a-b)(a-b) = a*a + b*b - 2ab} \\
    (t-2) &= \pm \sqrt{5} \\
    t &= \pm \sqrt{5} + 2 \\
\end{aligned}
\end{equation}  
So \eqref{eq:a4} has solutions: 
\begin{equation}  
\begin{aligned}
   r_1 &= \sqrt{5} + 2 \\
   r_2 &= -\sqrt{5} + 2  \\
\end{aligned}
\end{equation} 
\(r_0, r_1\) are all distincts.
Hence solution to LRS takes the following form:
\begin{equation} \label{eq:a5}
 a_n = a(r_1)^n + b(r_2)^n 
\end{equation} 
Solve \eqref{eq:a5}  for \(n=0\)
\begin{equation} 
\begin{aligned}
   a_0 &= a(r_1)^0 + b(r_2)^0  \\
   a_0 &= a + b  \\
   a &= a_0 - b   \\
   a &= G_0 - b \text{ since } a_0 = G_0 = 0 \\
   a &= -b
\end{aligned}
\end{equation} 
Solve \eqref{eq:a5}  for \(n=1\)
\begin{equation} 
\begin{aligned}
   a_1 &= a(r_1)^1 + b(r_2)^1  \\
   a_1 &= a(r_1) + b(r_2)  \\
   a_1 &= a(r_1) - a(r_2)   \\
   G_1 &= a ((r_1) - (r_2)) \text{ since } G_1 = a_1 = 2 \\
   2 &= a (2\sqrt{5} )\\
   a &= \frac{1}{\sqrt{5}}\\
\end{aligned}
\end{equation}     
So \eqref{eq:a5} has solutions:     
\(a = \frac{1}{\sqrt{5}}, b = -\frac{1}{\sqrt{5}}   \\
\) \\
So we have:
\begin{equation} \label{eq:a6}
 a_n = \frac{1}{\sqrt{5}}(\sqrt{5} + 2)^n + -\frac{1}{\sqrt{5}}(-\sqrt{5} + 2)^n 
\end{equation} 
Find biggest \(n\) such that \(a_n \leq 4000000 \) 
\begin{equation}
\begin{aligned} \label{eq:a7}
\frac{1}{\sqrt{5}}(2+\sqrt{5})^n - \frac{1}{\sqrt{5}}(2-\sqrt{5})^n &\leq 4000000 \\
(2+\sqrt{5})^n - (2 - \sqrt{5})^n &\leq 4000000\sqrt{5} \\
(2+\sqrt{5})^n  &\leq 4000000\sqrt{5} + (2 - \sqrt{5})^n\\
\lfloor (2+\sqrt{5})^n \rfloor &\leq \lfloor 4000000\sqrt{5} + (2 - \sqrt{5})^n \rfloor \\
\lfloor (2+\sqrt{5})^n \rfloor &\leq  4000000\sqrt{5} + \lfloor (2 - \sqrt{5})^n \rfloor \\
\lfloor (2+\sqrt{5})^n \rfloor &\leq  4000000\sqrt{5} \text{ since }  \lfloor (2 - \sqrt{5})^n \rfloor = \lfloor (-0.23...)^n \rfloor  = 0\\ 
\lfloor \log_{10}({(2+\sqrt{5})^n}) \rfloor &\leq log_{10}({4000000\sqrt{5}}) \\
\lfloor n \log_{10}{(2+\sqrt{5}}) \rfloor &\leq log_{10}({4000000\sqrt{5}}) \\
\lfloor n \rfloor &\leq \frac{log_{10}({4000000\sqrt{5}})}{\lfloor \log_{10}{(2+\sqrt{5}}) \rfloor} \\
\end{aligned}
\end{equation}
So we have: \\
\(n = \lfloor \frac{log_{10}({4000000\sqrt{5}})}{\log_{10}{(2+\sqrt{5}})} \rfloor \) \\ \\
Hence Sum \(S_n\) of even Fibonaci terms from 0 to \(n\) is:
\begin{equation} \label{eq:a8}
\begin{aligned}
 S_n &= a_0 + a_1 + a_2 + a_3 + ... + a_n \\
 S_n &=  \frac{1}{\sqrt{5}}(\sqrt{5} + 2)^0 + -\frac{1}{\sqrt{5}}(-\sqrt{5} + 2)^0 + ... +  \frac{1}{\sqrt{5}}(\sqrt{5} + 2)^n + -\frac{1}{\sqrt{5}}(-\sqrt{5} + 2)^n \\
 S_n &=  \frac{1}{\sqrt{5}}(\sqrt{5} + 2)^0 + ...+ \frac{1}{\sqrt{5}}(\sqrt{5} + 2)^n - (\frac{1}{\sqrt{5}}(-\sqrt{5} + 2)^n + ... +\frac{1}{\sqrt{5}}(-\sqrt{5} + 2)^0) \\
S_n &=  \sum_{k=1}^{n} \frac{1}{\sqrt{5}}(\sqrt{5} + 2)^k - (\sum_{k=1}^{n} \frac{1}{\sqrt{5}}(-\sqrt{5} + 2)^k) \\
\end{aligned}
\end{equation} 
We see that \(\sum_{k=1}^{n} \frac{1}{\sqrt{5}}(\sqrt{5} + 2)^k and (\sum_{k=1}^{n} \frac{1}{\sqrt{5}}(-\sqrt{5} + 2)^k)\) are two \href{http://en.wikipedia.org/wiki/Geometric_series}{Geometric Series} of the form:
\begin{equation} \label{eq:a9}
\begin{aligned}
\sum_{k=1}^{n} cx^k = cx+cx^2+ \cdots + cx^n= c\frac{x-x^{n+1}}{1-x}
\end{aligned}
\end{equation} 
Hence we have:
\begin{equation} \label{eq:a10}
\boxed{
 S_n = \sum_{k=1}^{n}  \frac{1}{\sqrt{5}}(2+\sqrt{5})^n-\frac{1}{\sqrt{5}}(2-\sqrt{5})^n = \displaystyle \cfrac{\cfrac{x-x^{n+1}}{1-x}-\cfrac{y-y^{n+1}}{1-y}}{\sqrt{5}} \\
}
\end{equation} 
with \(x=2+\sqrt{5}, y=2−\sqrt{5}, n = \lfloor log(L\sqrt{5})/log(2+\sqrt{5})\rfloor, L = 4000000\) 
\end{document}
